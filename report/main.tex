\documentclass{article}
\usepackage{graphicx}
\usepackage{xcolor}
\usepackage{hyperref}
\usepackage[bibstyle=authoryear, style =numeric, citestyle=numeric-comp]{biblatex}
\bibliography{bibliography/krr,bibliography/procs,bibliography/local}


\begin{document}

\title{Plan merging}

\author{Aurélien SIMON}

\maketitle

\begin{abstract}
We are talking about plan merging here
\end{abstract}


\newpage
\section{Definition}

Plan merging aim to solve MAPF~\cite{ststfekomawaliatcokubabo19a,erkiozsc13a} problems by computing individual plan for each agent to their goal regardless of conflict, and then, by using these previous plan, merge them into conflict-free plans. The task can be devided in three; target assigment, individual pathfinding and plan-merging.

\subsection{Classic MAPF}
We can base plan merging definition on classic MAPF's one; the paper ``Multi-Agent Pathfinding: Definitions, Variants, and Benchmarks''~\cite{ststfekomawaliatcokubabo19a} provides a standart defintion for non-anonymous MAPF with \(k\) agents\@. And it is defined as such; it takes as input a tuple \(\langle G,s,f \rangle \) where \(G=(V,E)\) represent a graph.  \(s : [1,\ldots,k] \rightarrow V\) maps an an agent to a  ``start'' vertex. Finally,  \(f : [1,\ldots,k] \rightarrow V\) maps an agent to a ``final'' vertex. The output is a set of \(k\) single-agent plan, where a plan \(\pi\) is denoted as a sequence of \(n\) action \(a_n,\ldots,a_1\), each of them being a function defined as \(a : V \rightarrow V\), they denote a movement from a vertex to another iff an edge exist between these two or a wating if the two vertices are the same. Formally, \(v \in V, v' \in V, \left\{ \begin{array}{ c l }
    \exists e(v,v') \in E \\
    v=v'              
  \end{array}
\right.\) \(\Rightarrow  a(v) = v'\). Considering a plan \(\pi\) for an agent \(r\) and a timestep \(t\), \(\pi_r[t] =a_t(a_{t-1}(\ldots a_1(s(r))))\). A plan is considered as valid iff \(\pi_r[|\pi_r-1|] = a_{|\pi_r|-1}(\ldots ( a_1(s(r)))) = f(r)\) where \(|\pi_r|\) gives the length of the plan \(\pi_r\). In order to have a valid solution, taken pairly, plans must be conflict-free;
\begin{enumerate}
    \item a vertex conflict between two agents \(r\) and \(r'\) occurs if, at timestep \(t\), \(\pi_r[t]=\pi_{r'}[t]\).
    \item an edge conflict  (also called swapping conflict) between two agents \(r\) and \(r'\) occurs if, at timestep \(t\), \(\pi_r[t]=\pi_{r'}[t-1]\) and \(\pi_r[t-1]=\pi_{r'}[t]\).

\end{enumerate} 


\subsection{Target assignment}
Basing ourselves on MAPF definition, we can then define target assignment (TA); it takes as input a tuple \(\langle G,s,F \rangle\), \(G=(V,E)\) representing a graph, \(s : [1,\ldots,k] \rightarrow V\) maps an agent to a ``start'' and \(F\) a set of \(k\) vertices denoting ``final'' vertices.  The output would be a tuple \(\langle G,s,f \rangle \) where \(G=(V,E)\) representing a graph, \(s : [1,\ldots,k] \rightarrow V\) maps an agent to a ``start'' vertex. Finally,  \(f : [1,\ldots,k] \rightarrow \{V\}\) maps an agent to a set of ``final'' vertices. Having a set of vertices as part of the output allows to describe both anonymous and non-anonymous MAPF problems, TA's output for anonymous variant would be the set of goal vertices \(F\) itself and a singleton for non-anonymous variant.

\subsection{Individual Pathfinding}
We would then define individual pathfinding (IP); the input of would be TA's output \(\langle G,s,f \rangle \). IP would then give as output, \(\langle G, \theta \rangle \) where each agent provides at least one solution. Formally, for each agent \(r\), we have \( \theta[r] = \{\pi_1,\ldots,\pi_n\}\). As a consequence of the TA's definition, let an agent \(r\), \(\forall \pi \in \theta[r], \pi[|\pi|-1]\in f(r)\). 

\subsection{Plan merging}
Finally, we can define plan merging (PM). It takes as input \(\langle G, \theta \rangle \) since we can deduce start and final position from plans. And gives as output MAPF's one: a solution being a set of \(k\) conflict-free plans. 

\printbibliography{}
\end{document}
